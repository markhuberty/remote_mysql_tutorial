%\documentclass[14pt]{extarticle}
\documentclass[11pt]{article}
\usepackage{natbib}
\usepackage[dvipdfm,colorlinks=true,urlcolor=DarkBlue,linkcolor=DarkBlue,bookmarks=false,citecolor=DarkBlue]{hyperref}

\usepackage[pdftex]{graphicx}
\usepackage{fancyhdr}
\usepackage[T1]{fontenc}
\usepackage{palatino}
\usepackage[utf8]{inputenc}
%\usepackage[super]{nth}
\usepackage{setspace}
\usepackage{placeins}
\usepackage{subfigure}
\usepackage{multirow}
\usepackage{rotating}
\usepackage{marvosym}  % Used for euro symbols with \EUR
\newcommand{\HRule}{\rule{\linewidth}{0.5mm}}
\usepackage{longtable} %% Allows the use of the longtable format produced by xl2latex.rb
\usepackage{lscape} %% Allows landscape orientation of tables
\usepackage{appendix} %% Allows customization of the appendix properties
\setcounter{tocdepth}{1} %% Restricts the table of contents to the section header level entries only

\usepackage{geometry}
\geometry{letterpaper}
\usepackage{amsmath}
\usepackage[stable]{footmisc}

%% The following settings are for the listings environment in R
\usepackage{listings}
\usepackage[svgnames]{xcolor}
\usepackage{soul}
\sethlcolor{LightGoldenrodYellow}
\lstset{backgroundcolor=\color{LightYellow}}
\lstset{framextopmargin=6pt, framexbottommargin=6pt, framerule=4pt,
  rulecolor=\color{White}, showspaces=false, showstringspaces=false}

\title{Remote access and processing of PATSTAT data}
\author{Mark Huberty\thanks{Travers Department of Political Science,
    University of California, Berkeley. Contact:
    \url{markhuberty@berkeley.edu}.}}
\date{\today}

\graphicspath{{../figures/}}

\begin{document}
%\input{/Users/markhuberty/Documents/Technology/Documentation/Typesetting/prospectusTitle.tex}
\maketitle
\doublespacing

\section{Introduction}
\label{sec:introduction}

This document provides background on how to access the PATSTAT
database as hosted at \texttt{durkheim.berkeley.edu}, use R and python
to query and process data, and move results back to your local machine
for analysis and use. It assumes that any access to \texttt{durkheim}
will occur via the command line, or via implicit remote connections
through a locally-invoked version of R, python, or MATLAB. 

All code for this session is hosted at \texttt{github}. See
\url{https://github.com/markhuberty/durkheim_mysql_remote} for more
detail. 


\section{Tools}
\label{sec:tools}

Remote access will require several tools. For the purposes of this
document, we assume you are using the following:


\begin{itemize}
\item Remote access
  \begin{itemize}
  \item Windows 

    Windows users need to install separate programs for
    remote
    access. PuTTY\footnote{\url{http://www.chiark.greenend.org.uk/~sgtatham/putty/}}
    is the default client and works well for most people. You should
    have both the SSH (remote shell access) and SCP (remote copy)
    pieces installed.
  \item OS X 

    OS X users have access to the SSH and SCP clients via the Terminal
    application found under \texttt{Applications/Utilities}.
  \item Linux 

    Linux users should have the SSH client installed by
    default and available inside their distribution's terminal emulator.
  \end{itemize}
\item Python

Python access to the database should occur using the \texttt{MySQLdb}
module\footnote{\url{http://mysql-python.sourceforge.net/}}. That is
installed on the server and can be called via \texttt{import MySQLdb}
or a suitable variant.
\item R

R access to the database is most easily had through the
\texttt{RMySQL}
interface\footnote{\url{http://cran.r-project.org/web/packages/RMySQL/index.html}},
available through CRAN.
\item MATLAB

Matlab access to MySQL and other databases is supported via the
\texttt{database}
routines\footnote{\url{http://www.mathworks.com/help/toolbox/database/ug/database.html}}. 
\end{itemize}

\section{Remote or local sessions?}
\label{sec:remote-or-local}

Database access can occur through either a remote or a local
session. ``Remote'' here means that the R or python process is running
on \texttt{durkheim} itself, so that the connection to the database
occurs from a process on the server to the server database. ``Local''
means that the R, python, or MATLAB process is running on a user's
local machine, which accesses \texttt{durkheim} solely for the
purpose of connecting to the database, executing the database query,
and pulling the data back into the user's local session. 

``\textbf{Local}'' access has clear advantages for ease of use. However, users
who want to do this should recognize that there are also downsides:
\begin{itemize}
\item If the network connection is unstable, it might die mid-query,
  taking the query and any output with it. This is particularly
  problematic for querying large databases like PATSTAT, where
  execution times might be minutes or tens of minutes
\item If the output is large, there will
  be a long delay as the output is pulled back through the network
  connection. So, for instance, a 1 Mb/s connection (a FAST
  connection) will take 1.5 minutes to pull back 100mb of output.
\end{itemize}
We recommend that users wishing to run local sessions connect to the
database via an SSH tunnel, which is more secure and easier for the
server administrator to manage. More information on tunnelling is
provided in each section below.


``\textbf{Remote}'' connections can work in one of three ways:
\begin{enumerate}
\item Log into the server and run an interactive session in R or
  python, pushing code stored on the server to an R or python session
  running on the server
\item Log into the server but run local code on the remote
  server. Note that this is distinct from the ``local'' option above:
  though the code is local, the R or python process is on the remote
  box, so there's no need to suck large amounts of data over a network connection
\item Log into the server and run code as a batch process, in which a
  script executes beginning-to-end without user interaction. 
\end{enumerate}

Note that option (2) really only works for R, via either R Studio's
remote server protocols or Emacs' \texttt{ess-remote}
mode.\footnote{ipython html notebook might be an answer here...}


\section{Example R sessions}
\label{sec:sample-r-session}

\subsection{Interactive remote R session, remote code}
\label{sec:remote-r-session-1}

Emacs and ESS provide a full-featured remote suite for writing and
executing R code on the server. The steps are:
\begin{enumerate}
\item Log into \texttt{durkheim} via ssh
\item Open the desired R code file in Emacs (\texttt{emacs mycode.R})
\item Split the buffer (\texttt{C-x 3})
\item Start the R session in the new buffer (\texttt{M-x R})
\end{enumerate}
A sample result is shown in figure
\ref{fig:sample-emacs-r-remote}. You will now be able to send code
from the R code file to the R shell using standard Emacs codebindings. All
saved output (figures and data) are written to the remote filesystem,
not the local computer. The syntax for issuing database queries is
shown in section \ref{sec:database-querying-r}

\subsection{Interactive remote R session, local code with \texttt{ess-remote}}
\label{sec:remote-r-session}

For running an interactive remote session from code local to your
personal machine, Emacs provides the \texttt{ess-remote} function. The
steps to run this go like this:

\begin{enumerate}
\item Open the desired code file in emacs on your local machine
\item Split the buffer as desired (e.g, \texttt{C-x 3} for
  side-by-side buffers)
\item In the new buffer, invoke an emacs-shell (\texttt{M-x shell})
\item From the shell, log into the server and start R from the comamnd
  line, as in:\footnote{http://www.r-bloggers.com/run-a-remote-r-session-in-emacs-emacs-ess-r-ssh/}
\begin{verbatim}
$ ssh username@durkheim.berkeley.edu
Welcome to Linux durkheim 2.6.32-37-generic 
#81-Ubuntu SMP Fri Dec 2 20:32:42 UTC 2011 x86_64 GNU/Linux
Ubuntu 10.04.1 LTS

Welcome to Ubuntu!
 * Documentation:  https://help.ubuntu.com/

77 packages can be updated.
76 updates are security updates.

Last login: Mon Apr 16 14:26:33 2012 
from airbears-136-152-132-25.airbears.berkeley.edu
username@durkheim:~/ R
R version 2.13.0 (2011-04-13)
Copyright (C) 2011 The R Foundation for Statistical Computing
ISBN 3-900051-07-0
Platform: x86_64-unknown-linux-gnu (64-bit)

R is free software and comes with ABSOLUTELY NO WARRANTY.
You are welcome to redistribute it under certain conditions.
Type 'license()' or 'licence()' for distribution details.

  Natural language support but running in an English locale

R is a collaborative project with many contributors.
Type 'contributors()' for more information and
'citation()' on how to cite R or R packages in publications.

Type 'demo()' for some demos, 'help()' for on-line help, or
'help.start()' for an HTML browser interface to help.
Type 'q()' to quit R.

> 
\end{verbatim}
\item Attach the remote R session to your local \texttt{ess} process
  (\texttt{M-x ess-remote})
\end{enumerate}

You should now be able to push code from your local file to the remote
R process with the standard ESS keybindings. Figure
\ref{fig:sample-emacs-r-remote} shows that this looks no different
from running in a local R session. Note, though, that any
\textit{saved} output (figures, data, etc) will be written to the
filesystem on the remote server. The syntax for issuing database
queries is shown in section \ref{sec:database-querying-r}. 

\subsection{Interactive local R session, remote database connection}
\label{sec:interactive-local-r}

Users who don't want to use Emacs and still want an interactive R
session may use whichever R interface they desire, and connect to
\texttt{durkheim} solely for interacting with the database. This
should be done through a
tunnelled SSH connection. This occurs in two steps: creating the
tunnel, and running the db connection in R.


To create the tunnel, log into \texttt{durkheim.berkeley.edu} using
an SSH client as follows:

\begin{verbatim}
ssh -L 3306:localhost:3306 username@durkheim.berkeley.edu
\end{verbatim}

This is doing three things:
\begin{enumerate}
\item Establishing an open SSH connection to the server
\item Forwarding port 3306 (which mysql listens on) through that connection
\item Mapping that port to localhost, the implicit address of the
  local machine
\end{enumerate}

Users can then issue calls to the database from within a local R
session as follows:

\begin{lstlisting}[language=r, numbers=none, frame=single]
  library(DBI)
  username = "markhuberty"
  password = "***"
  db = "patstatOct2011"

  my.query = "SELECT count(*) FROM tls206_person 
              WHERE person_ctry_code=="US""
  
  ## Establish the connection
  conn = dbConnect("MySQL",
                   username=username,
                   password=password,
                   dbname=db,
                   host="127.0.0.1",
                   port=3306
                   )
  ## Pass the query to the db 
  query = dbSendQuery(conn, my.query)
  
  ## Note that -1 fetches all records. Set it to a 
  ## 0 < N < infinity to fetch N rows.
  output = fetch(query, -1)

  dbDisconnect(conn)
\end{lstlisting}

\subsection{Running R code in batch }
\label{sec:running-r-code}

You can also run R code on the server in batch mode. Here, you provide
a complete R script that runs and saves output to one or more files,
and run the entire script at once. Given that scripts may take some
time to execute, it's important to do this such that you can log out
(or the connection can die) without stopping the batch process. 

The steps to do so go as follows:
\begin{enumerate}
\item Log into \texttt{durkheim} with \texttt{ssh}
\item Execute your script with this syntax: 
\begin{verbatim}
nohup R CMD BATCH mycode.R &
\end{verbatim}
\item Wait for the script to finish and write output to the files you indicated
\end{enumerate}

Here, \texttt{nohup} provides a ``no hangup'' signal to the shell, so
that it does not stop the batch process when you log out. \texttt{R
  CMD BATCH mycode.R} executes the code in batch mode. Finally,
\texttt{\&} pushes the process to the background and returns control
of the shell to the user. Any status, print, or error messages are
written to a \texttt{nohup.out} file. 

\subsection{Database querying in R}
\label{sec:database-querying-r}

Querying a database in R is straightforward when using the
\texttt{DBI} package, available from CRAN. The code segment below
shows how this can be done easily: 

\begin{lstlisting}[language=r, numbers=none, frame=single]
  library(DBI)
  username = "markhuberty"
  password = "***"
  db = "patstatOct2011"

  my.query = "SELECT count(*) FROM tls206_person 
              WHERE person_ctry_code=="US""
  
  ## Establish the connection
  conn = dbConnect("MySQL",
                   username=username,
                   password=password,
                   dbname=db
                   )
  ## Pass the query to the db 
  query = dbSendQuery(conn, my.query)
  
  ## Note that -1 fetches all records. Set it to a 
  ## 0 < N < infinity to fetch N rows.
  output = fetch(query, -1)

  dbDisconnect(conn)
\end{lstlisting}
\label{lst:db-query-r}

\begin{landscape}
  \begin{figure}[ht]
    \centering
    \includegraphics[height=0.75\textwidth]{sample_emacs_ess_remote_session}
    \caption{A remote R session running from local code via
      \texttt{ess-remote}. The user's code is on their local machine,
      but the R session is running inside an ssh connection to
      \texttt{durkheim}, via ess-remote. Code can be sent from the code buffer to the
      ESS shell via standard Emacs ess-mode keybindings. }
    \label{fig:sample-emacs-r-remote}
  \end{figure}
\end{landscape}


\begin{landscape}
  \begin{figure}[ht]
    \centering
    \includegraphics[height=0.75\textwidth]{sample_emacs_r_session}
    \caption{A remote R session running with remote code and a remote
      shell.. The user is logged in to
      \texttt{durkheim}. The left-hand buffer is an R code file
      open in Emacs. The right-hand buffer is an interactice ESS shell. Code can be sent from the code buffer to the
      ESS shell via standard Emacs ess-mode keybindings. }
    \label{fig:sample-emacs-r-remote}
  \end{figure}
\end{landscape}


\section{A sample Python session}

\subsection{Local interactive sessions in python}
\label{sec:local-inter-sess}

You can run a local python session and only connect to the server for
the purposes of issuing database queries and receiving output. The
easiest way to do this is via a \texttt{ssh} tunnel. This is easily
done via any SSH client, including PuTTY. 

From the client window, log into \texttt{durkheim.berkeley.edu} as
follows:

\begin{verbatim}
ssh -L 3306:localhost:3306 username@durkheim.berkeley.edu
\end{verbatim}

This is doing three things:
\begin{enumerate}
\item Establishing an open SSH connection to the server
\item Forwarding port 3306 (which mysql listens on) through that connection
\item Mapping that port to localhost, the implicit address of the
  local machine
\end{enumerate}

Basically, this means that when python goes looking for mysql, it
looks for it over the server connection rather than on the local
machine. For more information on port forwarding for MySQL on Windows,
see \url{http://www.codemastershawn.com/library/tutorial/ssh.tunnel.php}.

Once this is done, fire up python and import \texttt{MySQLdb}. The
syntax for issuing a query looks like this:

\begin{lstlisting}[language=python]
  import MySQLdb
  
  my_query = "SELECT count(*) FROM tls206_person 
              WHERE person_ctry_code=="US""
  
  ## Establish the db connection
  ## Note here: if you are running remote code in a remote
  ## python session, host="localhost" is correct. Otherwise,
  ## host="" is required so that the code goes looking for the right
  ## remote connection
  conn = MySQLdb.connect(host="127.0.0.1",
                         user="markhuberty" ,
                         passwd="***",
                         db="patstatOct2011"
                         )

  conn_cursor = conn.cursor()

  ## Execute the query
  conn_cursor.execute(my_query)

  ## Collect the output
  output = conn_cursor.fetchall()
  ## Shut down the 
  conn_cursor.close()
  conn.close()  
\end{lstlisting}

\subsection{Remote interactive sessions in python}
\label{sec:remote-inter-sess}

Unlike R, python lacks a simple mechanism for sending local code to a
python session on a remote server. This leaves only the remote/remote
option. The iPython HTML Notebook functionality may provide a way
around this, but is less useful for data-intensive operations

\subsubsection{Remote code and remote python session with emacs}
\label{sec:remote-code-remote}

Emacs and iPython provide a powerful interactive data analysis
interface and make working remotely much easier. The steps are:
\begin{enumerate}
\item Log into \texttt{durkheim} via ssh
\item Open the desired Python code file in Emacs
\item Split the buffer (\texttt{C-x 3})
\item Start the ipython session (\texttt{C-c !})
\end{enumerate}
A sample result is shown in figure
\ref{fig:sample-emacs-ipython-remote}. You will now be able to send code from the python file to the iPython
session directly using the standard iPython keybindings. Again, all
saved output (figures and data) are written to the remote filesystem,
not the local computer. 

\subsubsection{Local code and remote python session with ipython-notebook}
\label{sec:local-code-remote}

(Still working on this problem; ipython html notebook might be the
best way to proceed.))


\subsubsection{Running python code in batch}
\label{sec:running-python-code}

Like R, python scripts can be run start-to-finish without an
interactive session. Also like R, the code must be structured to run
without user input, and to write any desired output to files. 

The process for running python code in batch is much the same as for R::
\begin{enumerate}
\item Push your script to your server account via either \texttt{scp}
  or \texttt{git}
\item Log into \texttt{durkheim} with \texttt{ssh}
\item Execute your script with this syntax: 
\begin{verbatim}
nohup python mycode.py mycode.R &
\end{verbatim}
\item Wait for the script to finish and write output to the files you indicated
\end{enumerate}

Note that, unlike R, python will only write status and error messages
to \texttt{nohup.out} when the entire process is complete. 

\subsection{Querying the database from Python}
\label{sec:query-datab-from}

Python provides easy access to MySQL databases via the
\texttt{MySQLdb} module, which permits querying and extraction of data
from the database into standard Python objects. \texttt{MySQLdb}
provides for both local and remote database connections, via a
consistent syntax. The code below
provides a sample syntax for doing so with PATSTAT.

\begin{lstlisting}[language=python, numbers=none, frame=single]
  import MySQLdb
  
  my.query = "SELECT count(*) FROM tls206_person 
              WHERE person_ctry_code=="US""
  
  ## Establish the db connection
  ## Note here: if you are running remote code in a remote
  ## python session, host="localhost" is correct. Otherwise,
  ## host="" is required so that the code goes looking for the right
  ## remote connection
  conn = MySQLdb.connect(host="localhost",
                         user="markhuberty" ,
                         passwd="***",
                         db="patstatOct2011"
                         )

  conn_cursor = conn.cursor()

  ## Execute the query
  conn_cursor.execute(my.query)

  ## Collect the output
  output = conn_cursor.fetchall()
  ## Shut down the 
  conn_cursor.close()
  conn.close()
\end{lstlisting}


\begin{landscape}
  \begin{figure}[ht]
    \centering
    \includegraphics[height=0.75\textwidth]{sample_emacs_ipython_session}
    \caption{A remote python session. The user is logged in to
      \texttt{durkheim}. The left-hand buffer is a python code file
      open in Emacs. The right-hand buffer is a \texttt{ipython}
      interactive shell. Code can be sent from the code buffer to the
      python shell via standard Emacs python-mode keybindings. }
    \label{fig:sample-emacs-ipython-remote}
  \end{figure}
\end{landscape}


\section{A sample MATLAB session}
\label{sec:sample-matl-sess}

\begin{center}
  NOTE: the author is not a MATLAB user and has no MATLAB install to
  test this against. Caveat emptor, RTFM, etc. 
\end{center}

\begin{lstlisting}[language=matlab, numbers=none, frame=single]

conn = database('test_db','markhuberty','***','Vendor','MySQL',...
          'Server','durkheim.berkeley.edu')

query = exec(conn, my.query)
output = fetch(query, -1)
\end{lstlisting}


\end{document}